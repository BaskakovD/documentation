\section{Краткое описание темы индивидуальной работы}
Программа предназначена для классификации набора записей по набору классов, в
которых принадлежность записи определяется путем проверки набора правил.

\section{Диаграмма классов}
\MelImg{class_diagram}{Диаграмма классов.}

\section{Результаты профилирования}
На вход программы были поданы файлы с 10000 записями и наборами правил. Как
видно из рисунка \ref{img:profile_result}, наибольшее количество времени
программа проводит в функциях из библиотеки Qt5. Первая из них — ucstrncmp
используется для сравнения названий правил и записей.
\MelImg{profile_result}{Результаты профилирования программы.}

\section{Выводы по результатам профилирования}
Так как вышеуказанные функции занимают малую долю (меньше 40 \%) работы
программы, здесь нет необходимости в оптимизировании кода.

\subsection{Hello}
Hello

Numbered list
\begin{enumerate}
    \item one
    \item two
    \item three
        \begin{enumerate}
            \item\label{part31} three.one
            \item three.two \ref{part31}
        \end{enumerate}
\end{enumerate}

Unnumbered list
\begin{itemize}
    \item one
    \item two
    \item three
        \begin{itemize}
            \item three.one
        \end{itemize}
\end{itemize}


\section{Tables}
\LTXtable{\textwidth}{tables/table.tex}

\section{Hello}
Loremm
\section{Hello2}
Loremm2

\appendix
\MelAppend{Hello very long long long long long long long long long hahahahahhahahahhahahahah}
Hahaha

