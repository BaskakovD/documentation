% Размер шрифта — 14 pt, размер страницы - A4.
\documentclass[14pt,a4paper]{extarticle}

% Поля страницы.
\usepackage[left=3cm,right=1cm,top=2cm,bottom=2cm]{geometry}

% Кодировка документа.
\usepackage[utf8]{inputenc}
% Какие языки будут использоваться.
\usepackage[english,russian]{babel}

% Неразрывный пробел.
\DeclareUnicodeCharacter{00A0}{~}

% Использовать шрифт Times New Roman.
\usepackage{times}
% Кодировка шрифта - с поддержкой русского языка.
\usepackage[T2A]{fontenc}
% Для красивого отображения русских шрифтов.
\usepackage{pscyr}
% Полуторный интервал. Установит полуторный интервал везде, даже там, где не нужно.
%\linespread{1.3}
% Полуторный интервал только там, где нужно.
\usepackage{setspace}
\onehalfspacing
% Если вокруг формул будут большие отступы, раскомментировать.
%\usepackage[nodisplayskipstretch]{setspace}

% Отступ для параграфов.
\usepackage{indentfirst}

% Для формул.
\usepackage{amsmath}
\usepackage{amsfonts}
\usepackage{amssymb}

% Для вставки картинок.
\usepackage{graphicx}
% Путь, где будут лежать картинки.
\graphicspath{{images/}}
% Для позиционирования рисунков.
\usepackage{float}
% Для смены разделителя с ':' на '.'.
\RequirePackage{caption}
\DeclareCaptionLabelSeparator{melpoint}{. }
\captionsetup{justification=centering,labelsep=melpoint}
% Для смены "Рис." на "Рисунок".
\addto\captionsrussian{\renewcommand{\figurename}{Рисунок}}

% Макрос для вставки рисунка по центру.
% Первый параметр (необязательный) - ширина рисунка.
% Второй параметр - путь до рисунка.
% Третий параметр - подпись к рисунку.
\newcommand{\melimg}[3][\textwidth]
{
    \begin{figure}[H]
        \center{\includegraphics[scale=1,width=#1]{#2}}
        \caption{#3}
        \label{img:#2}
    \end{figure}
}

% Чтобы заголовки секций были не жирными. Также запрет переноса строк в заголовках.
\usepackage{titlesec}
\titleformat{\section}{\raggedright}{\thesection}{1em}{}

% Чтобы шрифт содержания был не жирным. Также добавляются точки в качестве заполнителя.
\usepackage{titletoc}
\titlecontents{section}
[0pt] % Левая граница.
{} % Код до.
{ % Для нумерованных секций.
    \thecontentslabel\enspace
}
{ % Для ненумерованных секций.
    \contentsmargin{0pt}\large
}
{ % Что-то связанное с фильтрацией.
    \titlerule*[.5pc]{.}\contentspage
}
[] % Код после.

% Чтобы можно было кликать по содержимому, а также перенос длинных заголовков в содержании.
\usepackage[linktocpage,hidelinks]{hyperref}

\begin{document}
\begin{titlepage}
	\newpage

	\begin{center}
        Министерство образования и науки Российской Федерации\\
        Федеральное государственное бюджетное образовательное учреждение\\
        высшего профессионального образования\\
        «Волгоградский государственный технический университет»
    \end{center}

	\vspace{2em} 

    \begin{tabular}{p{0.43\linewidth} l}
         & УТВЕРЖДАЮ \\
         & зав. кафедрой ПОАС \\
         & \_\_\_\_\_\_\_\_\_\_\_\_\_\_\_\_ Дворянкин А. М. \\
         & «\_\_\_\_» \_\_\_\_\_\_\_\_\_\_\_\_\_\_\_\_\_\_\_ 2015 г.
    \end{tabular}

    \vspace{4em}

    \begin{center}
        Надежность и качество програмнного обеспечения \\
        Разбиение Java программы на компоненты \\
        Оптимизация проекта
    \end{center}

    \vspace{4em}

    \begin{tabular}[t]{l l}
        СОГЛАСОВАНО: & РАЗРАБОТЧИК: \\
        Руководитель работы & Студент группы ПрИн-266 \\
        доценты кафедры ПОАС & \\
        \_\_\_\_\_\_\_\_\_\_\_\_\_\_\_\_\_\_\_ Сычев О. А. & \_\_\_\_\_\_\_\_\_\_\_\_\_\_\_\_ Мелихов А. В. \\
        «\_\_\_\_» \_\_\_\_\_\_\_\_\_\_\_\_\_\_\_ 2015 г. & «\_\_\_\_» \_\_\_\_\_\_\_\_\_\_\_\_\_\_\_ 2015 г. \\
        \\
        \\
        & НОРМОКОНТРОЛЛЕР: \\
        & \\
        & \_\_\_\_\_\_\_\_\_\_\_\_\_\_\_\_ Мамонтов Д. П. \\
        & «\_\_\_\_» \_\_\_\_\_\_\_\_\_\_\_\_\_\_\_ 2015 г.
    \end{tabular}

    \vspace{3em}

	\begin{center} 2015 \end{center} 
\end{titlepage}

\setcounter{page}{2}
\tableofcontents
\newpage

\section{Краткое описание темы индивидуальной работы}
Программа предназначена для классификации набора записей по набору классов, в 
которых принадлежность записи определяется путем проверки набора правил.

\section{Диаграмма классов}
\melimg{class_diagram}{Диаграмма классов.}

\section{Результаты профилирования}
На вход программы были поданы файлы с 10000 записями и наборами правил. Как видно из рисунка \ref{img:profile_result}, наибольшее количество времени программа проводит в функциях из библиотеки Qt5. Первая из них — ucstrncmp используется для сравнения названий правил и записей.
\melimg{profile_result}{Результаты профилирования программы.}

\section{Выводы по результатам профилирования}
Так как вышеуказанные функции занимают малую долю (меньше 40 \%) работы программы, здесь нет необходимости в оптимизировании кода.

\subsection{Hello}
Hello

\end{document}
