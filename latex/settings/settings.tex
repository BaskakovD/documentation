% Поля страницы.
\usepackage[left=3cm,right=1cm,top=2cm,bottom=2cm]{geometry}

% Кодировка документа.
\usepackage[utf8]{inputenc}
% Какие языки будут использоваться.
\usepackage[english,russian]{babel}

% Неразрывный пробел.
\DeclareUnicodeCharacter{00A0}{~}

% Кодировка шрифта - с поддержкой русского языка.
\usepackage[T2A]{fontenc}
% Для красивого отображения русских шрифтов.
\usepackage{pscyr}
% Полуторный интервал. Установит полуторный интервал везде, даже там, где не нужно.
%\linespread{1.3}
% Полуторный интервал только там, где нужно.
\usepackage{setspace}
\onehalfspacing
% Если вокруг формул будут большие отступы, раскомментировать.
%\usepackage[nodisplayskipstretch]{setspace}

% Отступ для параграфов.
\usepackage{indentfirst}

% Для формул.
\usepackage{amsmath}
\usepackage{amsfonts}
\usepackage{amssymb}

% Для вставки картинок.
\usepackage{graphicx}
% Для позиционирования рисунков.
\usepackage{float}
% Для смены разделителя с ':' на '.'.
\RequirePackage{caption}
\DeclareCaptionLabelSeparator{melpoint}{. }
\captionsetup{justification=centering,labelsep=melpoint}
% Для смены "Рис." на "Рисунок".
\addto\captionsrussian{\renewcommand{\figurename}{Рисунок}}

% Для того, чтобы не было висячих заголовков.
\usepackage{needspace}

% Чтобы заголовки секций были не жирными. Также запрет переноса строк в заголовках.
\usepackage{titlesec}
\titleformat{\section}{\needspace{4\baselineskip}\raggedright}{\thesection}{1em}{}
\titleformat{\subsection}{\raggedright}{\thesubsection}{1em}{}

% Чтобы можно было кликать по содержимому, а также перенос длинных заголовков в содержании.
\usepackage[linktocpage,hidelinks]{hyperref}

% Задание маркеров нумерованных списков в виде: 1. 1.1. и т. д.
\renewcommand{\labelenumi}{\arabic{enumi}.}
\renewcommand{\labelenumii}{\arabic{enumii}.}
\renewcommand{\labelenumiii}{\arabic{enumiii}.}

% Задание маркера ненумерованных списков.
\renewcommand\labelitemi{—}
\renewcommand\labelitemii{—}
\renewcommand\labelitemiii{—}

% Для таблиц на нескольких страницах.
\usepackage{tabularx}
\usepackage{longtable}
\usepackage{ltxtable}

% Для подстветки кода.
\usepackage{listings}
% Для отображения красной стрелки в месте оборачивания строки.
\usepackage{xcolor}
% Заворачивать длинные строки.
\lstset{
    frame=single,
    breaklines=true,
    postbreak=\raisebox{0ex}[0ex][0ex]{\ensuremath{\color{red}\hookrightarrow\space}}
}

% Для содержания.
\usepackage{tocloft}
% Ставить точки между названием содержания и страницей.
\renewcommand{\cftsecleader}{\cftdotfill{\cftdotsep}}
% Убрать жирные шрифты.
\renewcommand{\cftsecfont}{\mdseries}
\renewcommand{\cftsecpagefont}{\mdseries}
\renewcommand{\cfttoctitlefont}{\mdseries}

% Макрос для вставки рисунка по центру.
%\arg[opt] Ширина рисунка.
%\arg Путь до рисунка.
%\arg Подпись к рисунку.
\newcommand{\MelImg}[3][\textwidth]
{
    \begin{figure}[h]
        \center{\includegraphics[scale=1,width=#1]{#2}}
        \caption{#3}
        \label{img:#2}
    \end{figure}
}

% Для приложения.
%\arg Заголовок приложения.
%\arg Метка.
\newcommand{\MelAppend}[2]
{
    \newpage
    \refstepcounter{MelAppendixCounter}
    \label{app:#2}
    \section*{\hfillПриложение \theMelAppendixCounter}
    \begin{center}
        #1
    \end{center}
    \markboth{#1}{}
    \addcontentsline{toc}{section}{Приложение \theMelAppendixCounter. #1}
}

% Макрос для генерации содержания.
%\arg Номер страницы содержания.
\newcommand{\MelToc}[1]
{
    \ifnum#1>-1
        \setcounter{page}{#1}
    \fi
    \tableofcontents
    \newpage
}

% Макрос для вставки ненумерованной секции, которая будет отображаться в
% содержимом.
%\arg Заголовок секции.
\newcommand{\MelUnnumSec}[1]
{
    \section*{#1}
    \addcontentsline{toc}{section}{#1}
}



