% Макрос для вставки рисунка по центру.
%\arg[opt] Ширина рисунка.
%\arg Путь до рисунка.
%\arg Подпись к рисунку.
\newcommand{\MelImg}[3][\textwidth]
{
    \begin{figure}[h]
        \center{\includegraphics[scale=1,width=#1]{#2}}
        \caption{#3}
        \label{img:#2}
    \end{figure}
}

% Для приложения.
%\arg Заголовок приложения.
%\arg Метка.
\newcommand{\MelAppend}[2]
{
    \newpage
    \refstepcounter{MelAppendixCounter}
    \label{app:#2}
    \section*{\hfillПриложение \theMelAppendixCounter}
    \begin{center}
        #1
    \end{center}
    \markboth{#1}{}
    \addcontentsline{toc}{section}{Приложение \theMelAppendixCounter. #1}
}

% Макрос для генерации содержания.
%\arg Номер страницы содержания.
\newcommand{\MelToc}[1]
{
    \ifnum#1>-1
        \setcounter{page}{#1}
    \fi
    \tableofcontents
    \newpage
}

% Макрос для вставки ненумерованной секции, которая будет отображаться в
% содержимом.
%\arg Заголовок секции.
\newcommand{\MelUnnumSec}[1]
{
    \section*{#1}
    \addcontentsline{toc}{section}{#1}
}

